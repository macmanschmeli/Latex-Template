%%%%%%%%%%%%%%%%%%%%%%%%%%% asme2ej.tex %%%%%%%%%%%%%%%%%%%%%%%%%%%%%%%
% Template for producing ASME-format journal articles using LaTeX    %
% Written by   Harry H. Cheng, Professor and Director                %
%              Integration Engineering Laboratory                    %
%              Department of Mechanical and Aeronautical Engineering %
%              University of California                              %
%              Davis, CA 95616                                       %
%              Tel: (530) 752-5020 (office)                          %
%                   (530) 752-1028 (lab)                             %
%              Fax: (530) 752-4158                                   %
%              Email: hhcheng@ucdavis.edu                            %
%              WWW:   http://iel.ucdavis.edu/people/cheng.html       %
%              May 7, 1994                                           %
% Modified: February 16, 2001 by Harry H. Cheng                      %
% Modified: January  01, 2003 by Geoffrey R. Shiflett                %
% Use at your own risk, send complaints to /dev/null                 %
%%%%%%%%%%%%%%%%%%%%%%%%%%%%%%%%%%%%%%%%%%%%%%%%%%%%%%%%%%%%%%%%%%%%%%

%%% use twocolumn and 10pt options with the asme2ej format
\documentclass[twocolumn,10pt]{asme2ej}

\usepackage{graphicx} %% for loading jpg figures

%% The class has several options
%  onecolumn/twocolumn - format for one or two columns per page
%  10pt/11pt/12pt - use 10, 11, or 12 point font
%  oneside/twoside - format for oneside/twosided printing
%  final/draft - format for final/draft copy
%  cleanfoot - take out copyright info in footer leave page number
%  cleanhead - take out the conference banner on the title page
%  titlepage/notitlepage - put in titlepage or leave out titlepage
%  
%% The default is oneside, onecolumn, 10pt, final


\title{An ASME Journal Article Created Using 
\LaTeX2\raisebox{-.3ex}{$\epsilon$}
in ASME Format for Testing Your Figures}

%%% first author
\author{Harry H. Cheng
    \affiliation{
	Professor, Fellow of ASME\\
	Integration Engineering Laboratory\\
	Department of Mechanical Engineering\\
	University of California\\
	Davis, California 95616\\
    Email: hhcheng@ucdavis.edu
    }	
}

%%% second author
%%% remove the following entry for single author papers
%%% add more entries for additional authors
\author{J. Michael McCarthy\thanks{Address all correspondence related to ASME style format and figures to this author.} \\
    \affiliation{ Editor, Fellow of ASME\\
	Journal of Mechanical Design\\
        Email: jmmccart@uci.edu
    }
}

%%% third author
%%% remove the following entry for single author papers
%%% add more entries for additional authors
\author{Third Co-author\\
        Graduate Research Assistan, Student Member of ASME\\
       {\tensfb Fourth Co-author}\thanks{Address all correspondence for other issues to this author.} 
    \affiliation{Title, Member of ASME\\
        Department or Division Name\\
        Company or College Name\\
        City, State (spelled out), Zip Code\\
        Country (only if not U.S.)\\
        Email address (if available)
    }
}


\begin{document}

\maketitle    

%%%%%%%%%%%%%%%%%%%%%%%%%%%%%%%%%%%%%%%%%%%%%%%%%%%%%%%%%%%%%%%%%%%%%%
\begin{abstract}
{\it This is the abstract.
An abstract for an ASME paper should be less than 150 words and is normally in italics.
}
\end{abstract}

%%%%%%%%%%%%%%%%%%%%%%%%%%%%%%%%%%%%%%%%%%%%%%%%%%%%%%%%%%%%%%%%%%%%%%
\begin{nomenclature}
\entry{A}{You may include nomenclature here.}
\entry{$\alpha$}{There are two arguments for each entry of the nomemclature environment, the symbol and the definition.}
\end{nomenclature}

%%%%%%%%%%%%%%%%%%%%%%%%%%%%%%%%%%%%%%%%%%%%%%%%%%%%%%%%%%%%%%%%%%%%%%
\section{Introduction}

This article illustrates preparation of ASME paper using \LaTeX2\raisebox{-.3ex}{$\epsilon$}. The \LaTeX\  macro \verb+asme2ej.cls+, the {\sc Bib}\TeX\ style file \verb+asmems4.bst+, and the template \verb+asme2ej.tex+ that create this article are available on the WWW  at the URL address \verb+http://iel.ucdavis.edu/code/+. To ensure compliance with the 2003 ASME MS4 style guidelines  \cite{asmemanual}, you should modify neither the \LaTeX\ macro \verb+asme2ej.cls+ nor the {\sc Bib}\TeX\ style file \verb+asmems4.bst+. 

the given files are necessary and not to be included in the project
%%%%%%%%%%%%%%%%%%%%%%%%%%%%%%%%%%%%%%%%%%%%%%%%%%%%%%%%%%%%%%%%%%%%%%
\section{Footnotes\protect\footnotemark}
\footnotetext{Examine the input file, asme2ej.tex, to see how a footnote is given in a head.}

this is how one does Footnotes

%%%%%%%%%%%%%%%%%%%%%%%%%%%%%%%%%%%%%%%%%%%%%%%%%%%%%%%%%%%%%%%%%%%%%%
\section{Mathematics}

use this exact equation format including the enumeration
\begin{equation}
f(t) = \int_{0_+}^t F(t) dt + \frac{d g(t)}{d t}
\label{eq_ASME}
\end{equation}

%%%%%%%%%%%%%%%%%%%%%%%%%%%%%%%%%%%%%%%%%%%%%%%%%%%%%%%%%%%%%%%%%%%%%%
\section{Figures}
\label{sect_figure}

figures should be at the top and enumerated

%%%%%%%%%%%%%%%%%%%%%%%%%%%%%%%%%%%%%%%%%%%%%%%%%%%%%%%%%%%%%%%%%%%%%%
%%%%%%%%%%%%%%%% begin figure %%%%%%%%%%%%%%%%%%%
\begin{figure}[t]
\begin{center}
\setlength{\unitlength}{0.012500in}%
\begin{picture}(115,35)(255,545)
\thicklines
\put(255,545){\framebox(115,35){}}
\put(275,560){Beautiful Figure}
\end{picture}
\end{center}
\caption{The caption of a single sentence does not have period at the end}
\label{figure_ASME} 
\end{figure}
%%%%%%%%%%%%%%%% end figure %%%%%%%%%%%%%%%%%%% 
%%%%%%%%%%%%%%%%%%%%%%%%%%%%%%%%%%%%%%%%%%%%%%%%%%%%%%%%%%%%%%%%%%%%%%



%%%%%%%%%%%%%%%%%%%%%%%%%%%%%%%%%%%%%%%%%%%%%%%%%%%%%%%%%%%%%%%%%%%%%%
\section{Tables}

%%%%%%%%%%%%%%%%%%%%%%%%%%%%%%%%%%%%%%%%%%%%%%%%%%%%%%%%%%%%%%%%%%%%%%
%%%%%%%%%%%%%%% begin table   %%%%%%%%%%%%%%%%%%%%%%%%%%
\begin{table}[t]
\caption{Figure and table captions do not end with a period}
\begin{center}
\label{table_ASME}
\begin{tabular}{c l l}
& & \\ % put some space after the caption
\hline
Example & Time & Cost \\
\hline
1 & 12.5 & \$1,000 \\
2 & 24 & \$2,000 \\
\hline
\end{tabular}
\end{center}
\end{table}
%%%%%%%%%%%%%%%% end table %%%%%%%%%%%%%%%%%%% 
%%%%%%%%%%%%%%%%%%%%%%%%%%%%%%%%%%%%%%%%%%%%%%%%%%%%%%%%%%%%%%%%%%%%%%

All tables should be numbered consecutively  and centered above the table as shown in Table~\ref{table_ASME}. The body of the table should be no smaller than 7 pt.  There should be a minimum two line spaces between tables and text.


%%%%%%%%%%%%%%%%%%%%%%%%%%%%%%%%%%%%%%%%%%%%%%%%%%%%%%%%%%%%%%%%%%%%%%
\section{Citing References}

%%%%%%%%%%%%%%%%%%%%%%%%%%%%%%%%%%%%%%%%%%%%%%%%%%%%%%%%%%%%%%%%%%%%%%
The ASME reference format is defined in the authors kit provided by the ASME.  The format is:

\begin{quotation}
{\em Text Citation}. Within the text, references should be cited in  numerical order according to their order of appearance.  The numbered reference citation should be enclosed in brackets.
\end{quotation}


%%%%%%%%%%%%%%%%%%%%%%%%%%%%%%%%%%%%%%%%%%%%%%%%%%%%%%%%%%%%%%%%%%%%%%
\section{Conclusions}
put conclusion here

%%%%%%%%%%%%%%%%%%%%%%%%%%%%%%%%%%%%%%%%%%%%%%%%%%%%%%%%%%%%%%%%%%%%%%
\begin{acknowledgment}
put your acknowledgement here
\end{acknowledgment}

%%%%%%%%%%%%%%%%%%%%%%%%%%%%%%%%%%%%%%%%%%%%%%%%%%%%%%%%%%%%%%%%%%%%%%
% The bibliography is stored in an external database file
% in the BibTeX format (file_name.bib).  The bibliography is
% created by the following command and it will appear in this
% position in the document. You may, of course, create your
% own bibliography by using thebibliography environment as in
%
% \begin{thebibliography}{12}
% ...
% \bibitem{itemreference} D. E. Knudsen.
% {\em 1966 World Bnus Almanac.}
% {Permafrost Press, Novosibirsk.}
% ...
% \end{thebibliography}

% Here's where you specify the bibliography style file.
% The full file name for the bibliography style file 
% used for an ASME paper is asmems4.bst.
\bibliographystyle{asmems4}

% Here's where you specify the bibliography database file.
% The full file name of the bibliography database for this
% article is asme2e.bib. The name for your database is up
% to you.
\bibliography{asme2e}

%%%%%%%%%%%%%%%%%%%%%%%%%%%%%%%%%%%%%%%%%%%%%%%%%%%%%%%%%%%%%%%%%%%%%%
\appendix       %%% starting appendix
\section*{Appendix A: Head of First Appendix}
Avoid Appendices if possible.

\end{document}
